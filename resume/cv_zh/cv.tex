\documentclass{resume}
\usepackage[slantfont,boldfont,CJKnumber]{xeCJK}
\usepackage{bm}
\setCJKmainfont[BoldFont={SimHei}, ItalicFont={KaiTi}]{SimSun}
\setCJKsansfont{KaiTi}
\setCJKmonofont{STFangsong}


\setCJKfamilyfont{song}{SimSun}
\setCJKfamilyfont{kai}{KaiTi}
\setCJKfamilyfont{hei}{SimHei}
\setCJKfamilyfont{yao}{FZYaoTi}
\setCJKfamilyfont{hwkai}{STKaiti}
\setCJKfamilyfont{hwfs}{STFangsong}
\setCJKfamilyfont{hws}{STSong}
\setCJKfamilyfont{fs}{FangSong}
\setCJKfamilyfont{yahei}{SimHei}

\newcommand\song{\CJKfamily{song}}
\newcommand\kai{\CJKfamily{kai}}
\newcommand\hei{\CJKfamily{hei}}
\newcommand\yao{\CJKfamily{yao}}
\newcommand\hwkai{\CJKfamily{hwkai}}
\newcommand\hwfs{\CJKfamily{hwfs}}
\newcommand\hws{\CJKfamily{hws}}
\newcommand\fs{\CJKfamily{fs}}
\newcommand\yahei{\CJKfamily{yahei}}

\newcommand{\erhao}{\fontsize{22pt}{\baselineskip}\selectfont}
\newcommand{\xiaoerhao}{\fontsize{18pt}{\baselineskip}\selectfont}
\newcommand{\sanhao}{\fontsize{16pt}{\baselineskip}\selectfont}
\newcommand{\xiaosanhao}{\fontsize{15pt}{\baselineskip}\selectfont}
\newcommand{\sihao}{\fontsize{14pt}{\baselineskip}\selectfont}
\newcommand{\xiaosihao}{\fontsize{12pt}{\baselineskip}\selectfont}
\newcommand{\wuhao}{\fontsize{10.5pt}{\baselineskip}\selectfont}
\newcommand{\xiaowuhao}{\fontsize{9pt}{\baselineskip}\selectfont}
\newcommand{\liuhao}{\fontsize{7.5pt}{\baselineskip}\selectfont}

\newcommand{\MyName}[1]{ % Name
		\sanhao \usefont{OT1}{phv}{b}{n} \hfill #1
		\par \normalsize \normalfont}
\newcommand{\MySlogan}[1]{ % Slogan (optional)
		\wuhao \usefont{OT1}{phv}{m}{n}\hfill #1
		\par \normalsize \normalfont}

\usepackage[left=0.75in,top=0.6in,right=0.75in,bottom=0.6in]{geometry} % Document margins
\usepackage[colorlinks, citecolor=red]{hyperref}

\name{\erhao\MyName{丁铖}} % Your name
%\address{\MySlogan{中山北路3663号,华东师范大学,200062}}
\address{\MySlogan{(+86)185 1621 6459}} % Your phone number and email
\address{\MySlogan{teisei.din@gmail.com}}

%\name{\erhao\MyName{丁铖}} % Your name
%\address{\MySlogan{中山北路3663号,华东师范大学,200062}}
%\address{\MySlogan{(+86)185 1621 6459}} % Your phone number and email
%\address{\MySlogan{teisei.din@gmail.com}}


\begin{document}

%-----------------------------------------------------------------
%---------------------------意向----------------------------------
%---------------------------意向----------------------------------
%-----------------------------------------------------------------

%\begin{rSection}{\fs \sanhao {意向}}
%\sihao \textit{软件工程师—— 金融服务系统、应用开发、数据分析}\\
%\sihao \textit{软件工程师——金融服务系统、应用开发,数据分析}
%\end{rSection}

%-----------------------------------------------------------------
%-------------------------- 教育经历------------------------------
%---------------------------教育经历------------------------------
%-----------------------------------------------------------------

\begin{rSection}{\fs \sanhao {教育经历}}
\sihao \textbf {硕士,华东师范大学,\href{http://database.ecnu.edu.cn}{海量计算研究所}} \hfill {\em 2014,09-2017,06}\\
\sihao \textit{导师:\href{https://wnqian.wordpress.com/}{钱卫宁}}   
\sihao \textit{研究方向:大数据分析,社交媒体分析}\\
%\sihao \textit{课程:软件理论基础,算法设计与分析,数据库理论,高级信息检索,数据管理与实现,计算广告学,大数据处理方法,系统体系结构,海量数据分析与挖掘 }\\
\\
\sihao \textbf {本科, 大连海事大学,信息科学与技术学院} \hfill {\em 2010,09-2014,06}\\
\sihao \textit{专业:软件工程(日语强化)}
\sihao \textit{综合绩点:}\normalfont{3.62/4.0 }  \\
\sihao \textit{主修课程:数据结构,算法设计}\\


\end{rSection}

%-----------------------------------------------------------------
%-------------------------- 项目经历------------------------------
%---------------------------项目经历------------------------------
%-----------------------------------------------------------------


%-----------------------------------------------------------------
%-------------------------- 实习经历------------------------------
%---------------------------实习经历------------------------------
%-----------------------------------------------------------------

\begin{rSection}{\fs \sanhao {实习经历}}
	\sihao \textbf{Splunk上海研发中心}  \hfill {\em 2016,07 - 至今}{ 软件工程师}{}\\
	\sihao \textit{    
		开发一个可以对多个Splunk集群进行实时性能监控和行为分析的Splunk应用。}\\
	\sihao \textit{    相关技术:Splunk知识, Python, 日志抓取,RESTful, 分布式系统。}\\
	\\
	\sihao \textbf{Infosys Labs,班加罗尔}  \hfill {\em 2015,10-2015,12}{ 研发工程师}{}\\
	\sihao \textit{    根据员工信息和项目分配信息构建企业社交网络,开发Restful Web Service,提供员工、项目的搜索、推荐功能;研究针对项目需求的团队组建问题。}\\
	\sihao \textit{    相关技术:Java,Neo4j,Lucene,RESTful service,R ,Hadoop ,MapReduce,Zookeeper}
	\\
\end{rSection}



\begin{rSection}{\fs \sanhao {项目经历}}
%海量web数据内容管理、分析挖掘技术与大型示范应用
\sihao \textbf {    海量web数据内容管理、分析(863国家项目子课题)} \hfill {\em 2015,03-2015,05}\\
$\bullet$ \textit{   将从不同网站爬取的GB级别的新闻数据进行整合、存储、索引;}\\
$\bullet$ \textit{   用自然语言处理的方法从数据集中抽取公司名、公司之间关系;}\\
$\bullet$ \textit{   相关技术:Java, 自然语言处理,Lucene,MySQL。}\\
\\
\\
%---------------------------------------------------------------------------------------
%---------------------------------------------------------------------------------------
\sihao \textbf {海量集群行为数据的聚融与管理(973国家项目子课题)} \hfill {\em 2014,11-2015,01}\\
$\bullet$ \textit{  开发分布式爬虫爬取TB级别的微博数据,并将数据存储于HDFS;}\\
$\bullet$ \textit{  编写MapReduce程序处理微博数据;}\\
$\bullet$ \textit{  相关技术:Java, 微博API, Hadoop, MapReduce, Kafka。}\\
\\
%---------------------------------------------------------------------------------------
%---------------------------------------------------------------------------------------
%\sihao \textbf { MIPS系统模拟器} \hfill {\em 2013,07-2014,05}\\
%$\bullet$ \textit{  模拟MIPS CPU来执行一段合法程序指令段;}\\
%$\bullet$ \textit{   运用了面向对象的编程方法。}\\
%\\
%---------------------------------------------------------------------------------------
%---------------------------------------------------------------------------------------
%\sihao \textbf {      拼写检错器} \hfill {\em 2014,10-2014,11}\\
%$\bullet$ \textit{    编写基于词典, 编辑距离和k-gram索引的拼写纠错器;}\\
%$\bullet$ \textit{    相关技术: TrieTree, HMM, Lucene, Java。}\\
%\\
%---------------------------------------------------------------------------------------
%---------------------------------------------------------------------------------------
%\sihao \textbf {      校园失物招领系统设计与开发} \hfill {\em 2013,07-2014,05}\\
%$\bullet$ \textit {     设计与开发一个帮助觅物的人来寻找丢失的物品;}\\
%$\bullet$\textit {      训练了一个用于匹配失主和物品的向量空间模型。}\\
%\\
%---------------------------------------------------------------------------------------
%---------------------------------------------------------------------------------------
%\sihao \textbf {     安卓项目} \hfill {\em 2013,02-2013,04}\\
%$\bullet$ \textit{   一个通过模拟PC的键盘操作来远程控制PC的安卓APP。}\\
%$\bullet$ \textit{   界面类似BBC News的校园新闻APP。}\\
%$\bullet$ \textit{   相关技术: Android开发, 多线程, Java, 安卓传感器。}\\
%\\
\end{rSection}




%\begin{rSection}{\fs \sanhao {论文情况}}
%\begin{itemize}
%\item {\bf 软件著作权:微博事件实时分析系统.}
%\item {\bf 专利:一种社交媒体中热点微博数据的自适应取样方法.}
%\item {\bf 基于地图匹配的时空轨迹匿名算法.} (第三作者). {\em  中国计算机网络与信息安全学术会议 2013.}
%\item {\bf  面向社交数据流连续查询的基准评测.} (第一作者). {\em   华师大学报《内存计算数据管理》专刊 2014.}
%\item {\bf BSMA: A Benchmark for Analytical Queries over Socail Media Data.} (第二作者). {\em VLDB(Demo) 2014.}
%\item {\bf Sampling Social Streams for Hot Social Events Analytics.} (第一作者). {\em SSEPM(International Workshop on Scalable Social Event Processing and Management) ICDE 2015.}
%\end{itemize}
%\end{rSection}


%-----------------------------------------------------------------
%-------------------------- 荣誉证书------------------------------
%---------------------------荣誉证书------------------------------
%-----------------------------------------------------------------

\begin{rSection}{\fs \sanhao {荣誉与证书}}
	\sihao \textit{触宝科技2016黑客马拉松第二名} \hfill {\em 2016} \\
	%\sihao \textit{大连海事大学优秀毕业生} \hfill {\em 2014} \\
	\sihao \textit{微软编程之美2016复赛第36名(前10\%)} \hfill {\em 2016} \\
	\sihao \textit{国际大学生程序设计大赛中俄挑战赛, 第21名} \hfill {\em 2013} \\
	\sihao \textit{优秀学生奖学金(前5\%)} \hfill {\em 2011-2014} \\
\end{rSection}


%-----------------------------------------------------------------
%-------------------------- 专业技能------------------------------
%---------------------------专业技能------------------------------
%-----------------------------------------------------------------


\begin{rSection}{\fs \sanhao {技能}}
	\sihao \textbf {专业技能}\\
	\sihao \textit{ 熟练掌握:数据结构与算法设计, JAVA, Hadoop与MapReduce;\\
		基本掌握: Git, Linux, Python, C, RESTful, Lucene, Kafka, Zookeeper, Airflow等。}\\
	
	\sihao \textbf {外语能力}\\
	\sihao \textit{CET-6}\\
	\sihao \textit{国际日本语能力测试N2}\\
\end{rSection}


\begin{rSection}{\fs \sanhao {社会活动}}
\sihao \textit{ 野生动物保护宣传活动} \hfill {\em 2015, 印度班加罗尔} \\
\sihao \textit{CIEE国际学生中文辅导} \hfill {\em 2014,中国上海} \\
\sihao \textit{VLDB暑期学校志愿者} \hfill {\em 2014, 中国上海} \\
\sihao \textit{ Cookpad tech talk} \hfill {\em 2013, 日本东京} 
\end{rSection}

\begin{rSection}{\fs \sanhao { 兴趣爱好}}
	\sihao \textit{ 解题,开源,软件工程,烹饪,运动,旅行,摄影,志愿者} \hfill \\
\end{rSection}

\end{document} 